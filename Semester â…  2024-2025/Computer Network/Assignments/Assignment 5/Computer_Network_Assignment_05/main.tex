%!TeX program = xelatex
\documentclass[12pt,hyperref,a4paper,UTF8]{ctexart}
\usepackage{homework}
\usepackage{booktabs}
\usepackage{enumitem}

%%-------------------------------正文开始---------------------------%%
\begin{document}

%%-----------------------封面--------------------%%
\cover

%%------------------摘要-------------%%
%\begin{abstract}
%
%在此填写摘要内容
%
%\end{abstract}

\thispagestyle{empty} % 首页不显示页码

%%--------------------------目录页------------------------%%
\newpage
\tableofcontents

%%------------------------正文页从这里开始-------------------%
\newpage

%%可选择这里也放一个标题
%\begin{center}
%    \title{ \Huge \textbf{{标题}}}
%\end{center}

% ==============
% =    规则    =
% ==============
\section{规则}
\begin{enumerate}[label=\Roman*]
    \item DDL为每周日的23:59;

    \item 每次课前,评得示范作业的同学会给出自己的标准答案。如果对评分有不同看法,欢迎在听完示范作业之后,当面提出异议;
    
    \item 在规定时间内不能提交作业者,零分;

    \item 没有推导过程,只列出答案者,零分;

    \item 格式混乱,无法阅读者,零分;

    \item 照抄答案者,零分;

    \item 无法使用LaTeX完成作业者,零分;

    \item 不懂就问,不会就学。任何经验都是后天积累的;

    \item 关于LaTeX的一切,国内外的网站都有详尽的教学视频。例:
    \begin{itemize}
        \item \href{https://www.bilibili.com/video/BV1Jy4y1p76e/?vd_source=2c0f6624843da61e86c7e8a2b75de875}{Bilibili}

        \item \href{https://www.youtube.com/watch?v=Jp0lPj2-DQA&list=PLHXZ9OQGMqxcWWkx2DMnQmj5os2X5ZR73}{YouTube}
    \end{itemize}
\end{enumerate}

\newpage

% ==============
% =    习题    =
% ==============
\section{习题}

\subsection{ARP-1}
\textbf{问:}
\begin{enumerate}[label=\Roman*,leftmargin=2.2\parindent]
    \item 试解释为什么ARP高速缓存每存入一个项目就要设置10$\sim$20分钟的超时计时器。这个时间设置得太大或太小会出现什么问题?

    \item 至少举出两种不需要发送ARP 请求分组的情况(即不需要请求将某个目的IP地址解析为相应的MAC地址)。
\end{enumerate}

\textbf{答:}
TODO

\subsection{ARP-2}
\textbf{问:}
主机A发送IP数据报给主机B,途中经过了5 个路由器。试问在IP数据报的发送过程中总共使用了几次ARP?

\textbf{答:}
TODO

\subsection{路由器转发表}
\textbf{问:}
设某路由器建立了转发表\ref{tab:转发表}:
\begin{table}[h!]
    \centering
    \renewcommand{\arraystretch}{1}
    \setlength{\tabcolsep}{30pt}
    \begin{tabular}{l|r}
    \toprule
    前缀匹配    &   下一跳\\
    \midrule
    192.4.153.0/26  &   R3\\
    128.96.39.0/25  &   接口$m_0$\\
    128.96.39.128/25    &   接口$m_1$\\
    128.96.40.0 /25 &   $R_2$\\
    192.4.153.0/26  &   $R_3$\\
    *(默认)   &   $R_4$\\
    \bottomrule
    \end{tabular}
    \caption{转发表}
    \label{tab:转发表}
\end{table}

现共收到5个分组,其目的地址分别为:
\begin{enumerate}[label=(\arabic*),leftmargin=2.2\parindent]
    \item 128.96.39.10
    \item 128.96.40.12
    \item 128.96.40.151
    \item 192.4.153.17
    \item 192.4.153.90
\end{enumerate}

试分别计算其下一跳。

\textbf{答:}
TODO

\subsection{最大/最小IP}
\textbf{问:}
某单位分配到一个地址块$129.250/16$。该单位有$4000$台计算机,平均分布在$16$个不同的地点。试给每一个地点分配一个地址块,并算出每个地址块中IP地址的最小值和最大值。

\textbf{答:}
TODO

\subsection{数据报}
\textbf{问:}
一个数据报长度为$4000$字节(固定首部长度)。现在经过一个网络传送,但此网
络能够传送的最大数据长度为$1500$字节。试问应当划分为几个短些的数据报片?各数据报片的数据字段长度、片偏移字段和MF标志应为何数值?

\textbf{答:}
TODO

\end{document}