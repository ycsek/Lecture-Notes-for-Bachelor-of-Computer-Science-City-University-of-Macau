%!TeX program = xelatex
\documentclass[12pt,hyperref,a4paper,UTF8]{ctexart}
\usepackage{homework}
\usepackage{booktabs}
\usepackage{enumitem}

%%-------------------------------正文开始---------------------------%%
\begin{document}

%%-----------------------封面--------------------%%
\cover

%%------------------摘要-------------%%
%\begin{abstract}
%
%在此填写摘要内容
%
%\end{abstract}

\thispagestyle{empty} % 首页不显示页码

%%--------------------------目录页------------------------%%
\newpage
\tableofcontents

%%------------------------正文页从这里开始-------------------%
\newpage

%%可选择这里也放一个标题
%\begin{center}
%    \title{ \Huge \textbf{{标题}}}
%\end{center}

% ==============
% =    规则    =
% ==============
\section{规则}
\begin{enumerate}[label=\Roman*]
    \item DDL为每周日的23:59;

    \item 每次课前,评得示范作业的同学会给出自己的标准答案。如果对评分有不同看法,欢迎在听完示范作业之后,当面提出异议;
    
    \item 在规定时间内不能提交作业者,零分;

    \item 没有推导过程,只列出答案者,零分;

    \item 格式混乱,无法阅读者,零分;

    \item 照抄答案者,零分;

    \item 无法使用LaTeX完成作业者,零分;

    \item 不懂就问,不会就学。任何经验都是后天积累的;

    \item 关于LaTeX的一切,国内外的网站都有详尽的教学视频。例:
    \begin{itemize}
        \item \href{https://www.bilibili.com/video/BV1Jy4y1p76e/?vd_source=2c0f6624843da61e86c7e8a2b75de875}{Bilibili}

        \item \href{https://www.youtube.com/watch?v=Jp0lPj2-DQA&list=PLHXZ9OQGMqxcWWkx2DMnQmj5os2X5ZR73}{YouTube}
    \end{itemize}
\end{enumerate}

\newpage

% ==============
% =    习题    =
% ==============
\section{习题}

\subsection{地址前缀匹配}
\textbf{问:}
以下地址前缀中的哪一个地址与$2.52.90.140$ 匹配?请说明理由。
\begin{enumerate}[label=\Roman*),leftmargin=2.2\parindent]
    \item 0/4; 
    \item 32/4; 
    \item 4/6; 
    \item 80/4。
\end{enumerate}

\textbf{答:}
TODO

\subsection{掩码与网络前缀}
\textbf{问:}
与下列掩码相对应的网络前缀各有多少位?
\begin{enumerate}[label=\Roman*),leftmargin=2.2\parindent]
    \item $192.0.0.0$;
    \item $240.0.0.0$;
    \item $255.224.0.0$;
    \item $255.255.255.252$。
\end{enumerate}

\textbf{答:}
TODO

\subsection{路由表更新}
\textbf{问:}
假定网络中的路由器B的路由表如表\ref{tab:router_b}所示:
\begin{table}[h!]
    \centering
    \caption{路由器B的路由表}
    \begin{tabular}{c|c|c}
    \toprule
    目的网络 & 距离 & 下一跳路由器\\
    \midrule
    $N_1$ & $7$ & $A$\\
    $N_2$ & $2$ & $C$\\
    $N_6$ & $8$ & $F$\\
    $N_8$ & $4$ & $E$\\
    $N_9$ & $4$ & $F$\\
    \bottomrule
    \end{tabular}
    \label{tab:router_b}
\end{table}

现在B收到从C发来的路由信息,如表\ref{tab:updates}所示:
\begin{table}[h!]
    \centering
    \caption{路由器C至路由器B的更新信息}
    \begin{tabular}{c|c}
    \toprule
    目的网络 & 距离\\
    \midrule
    $N_2$ & $4$\\
    $N_3$ & $8$\\
    $N_6$ & $4$\\
    $N_8$ & $3$\\
    $N_9$ & $5$\\
    \bottomrule
    \end{tabular}
    \label{tab:updates}
\end{table}

试求出路由器B更新后的路由表(详细说明每一个步骤)。

\textbf{答:}
TODO

\subsection{IPv4地址转换}
\textbf{问:}
试把下列IPv4 地址从二进制记法转换为点分十进制记法:
\begin{enumerate}[label=\Roman*)]
    \item $10000001 \ 00001011 \ 00001011 \ 11101111$
    \item $11000001 \ 10000011 \ 00011011 \ 11111111$
    \item $11100111 \ 11011011 \ 10001011 \ 01101111$
    \item $11111001 \ 10011011 \ 11111011 \ 00001111$
\end{enumerate}

\textbf{答:}
TODO

\subsection{IPv4过渡至IPv6}
\textbf{问:}
从IPv4过渡到IPv6的方法有哪些?

\textbf{答:}
TODO

\end{document}