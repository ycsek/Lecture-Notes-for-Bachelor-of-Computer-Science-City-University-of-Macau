%!TeX program = xelatex
\documentclass[12pt,hyperref,a4paper,UTF8]{ctexart}
\usepackage{homework}

%%-------------------------------正文开始---------------------------%%
\begin{document}

%%-----------------------封面--------------------%%
\cover

%%------------------摘要-------------%%
%\begin{abstract}
%
%在此填写摘要内容
%
%\end{abstract}

\thispagestyle{empty} % 首页不显示页码

%%--------------------------目录页------------------------%%
\newpage
\tableofcontents

%%------------------------正文页从这里开始-------------------%
\newpage

%%可选择这里也放一个标题
%\begin{center}
%    \title{ \Huge \textbf{{标题}}}
%\end{center}

% ==============
% =    规则    =
% ==============
\section{规则}
\begin{enumerate}[I]
    \item 在规定时间内不能提交作业者,零分;

    \item 没有推导过程,只列出答案者,零分;

    \item 格式混乱,无法阅读者,零分;

    \item 不懂就问,不会就学。任何经验都是后天积累的。

    \item 关于LaTeX的一切,国内外的网站都有详尽的教学视频。例:
    \begin{itemize}
        \item \href{https://www.bilibili.com/video/BV1Jy4y1p76e/?vd_source=2c0f6624843da61e86c7e8a2b75de875}{Bilibili}

        \item \href{https://www.youtube.com/watch?v=Jp0lPj2-DQA&list=PLHXZ9OQGMqxcWWkx2DMnQmj5os2X5ZR73}{YouTube}
    \end{itemize}
\end{enumerate}

\newpage

% ==============
% =    习题    =
% ==============
\section{习题}

\subsection{物理层的概念}
\textbf{问:}物理层要解决哪些问题?物理层的主要特点是什么?

\textbf{答:}TODO

\subsection{物理层的接口}
\textbf{问:}物理层的接口有哪几个方面的特性?各包含什么内容?

\textbf{答:}TODO

\subsection{码元与振幅调制}
\textbf{问:}假定某信道受奈氏准则限制的最高码元速率为$20,000$码元/秒。如果采用振幅调制,把码元的振幅划分为16个不同等级来传送,那么可以获得多高的数据率(bit/s)?

\textbf{答:}TODO
\begin{equation}\label{eq:2.3}
\begin{aligned}
    E = mc^2
\end{aligned}
\end{equation}

\subsection{信号衰减}
\textbf{问:}假定有一种双绞线的衰减是$0.7$dB/km(在1kHz时),若容许有$20$dB的衰减,试问
\begin{itemize}
    \item 使用这种双绞线的链路的工作距离有多长?

    \item 如果要使这种双绞线的工作距离增大到$100$km,问应当使衰减降低到多少?
\end{itemize}


\textbf{答:}TODO
\begin{equation}\label{eq:2.4}
\begin{aligned}
    E = mc^2
\end{aligned}
\end{equation}

\subsection{信道复用}
\textbf{问:}为什么要使用信道复用技术?常用的信道复用技术有哪些?

\textbf{答:}TODO

\subsection{CDMA}
\textbf{问:}共有$4$个站进行码分多址CDMA通信,四个站的码片序列如\autoref{tab:cdma}所示。
\begin{table}[t!]
    \centering
    \begin{tabular}{cc}
       A: $(-1, -1, -1, +1, +1, -1, +1, +1)$  & B: $(-1, -1, +1, -1, +1, +1, +1, -1)$ \\
       C: $(-1, +1, -1, +1, +1, +1, -1, -1)$  & D: $(-1, +1, -1, -1, -1, -1, +1, -1)$
    \end{tabular}
    \caption{码片序列}
    \label{tab:cdma}
\end{table}
现收到这样的码片序列:$(-1, +1, -3, +1, -1, -3, +1, +1)$。
问
\begin{itemize}
    \item 哪个站发送数据了?
    
    \item 发送数据的站发送的是$1$还是$0$?
\end{itemize}


\textbf{答:}TODO
\begin{equation}\label{eq:2.6}
\begin{aligned}
    E = mc^2
\end{aligned}
\end{equation}



\end{document}