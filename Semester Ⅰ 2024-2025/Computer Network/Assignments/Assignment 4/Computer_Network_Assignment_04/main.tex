%!TeX program = xelatex
\documentclass[12pt,hyperref,a4paper,UTF8]{ctexart}
\usepackage{homework}
\usepackage{booktabs}

%%-------------------------------正文开始---------------------------%%
\begin{document}

%%-----------------------封面--------------------%%
\cover

%%------------------摘要-------------%%
%\begin{abstract}
%
%在此填写摘要内容
%
%\end{abstract}

\thispagestyle{empty} % 首页不显示页码

%%--------------------------目录页------------------------%%
\newpage
\tableofcontents

%%------------------------正文页从这里开始-------------------%
\newpage

%%可选择这里也放一个标题
%\begin{center}
%    \title{ \Huge \textbf{{标题}}}
%\end{center}

% ==============
% =    规则    =
% ==============
\section{规则}
\begin{enumerate}[I]
    \item DDL为每周日的23:59;
    
    \item 在规定时间内不能提交作业者,零分;

    \item 没有推导过程,只列出答案者,零分;

    \item 格式混乱,无法阅读者,零分;

    \item 照抄答案者,零分;

    \item 不懂就问,不会就学。任何经验都是后天积累的;

    \item 关于LaTeX的一切,国内外的网站都有详尽的教学视频。例:
    \begin{itemize}
        \item \href{https://www.bilibili.com/video/BV1Jy4y1p76e/?vd_source=2c0f6624843da61e86c7e8a2b75de875}{Bilibili}

        \item \href{https://www.youtube.com/watch?v=Jp0lPj2-DQA&list=PLHXZ9OQGMqxcWWkx2DMnQmj5os2X5ZR73}{YouTube}
    \end{itemize}
\end{enumerate}

\newpage

% ==============
% =    习题    =
% ==============
\section{习题}

\subsection{IP地址表示}
\textbf{问:}
IP地址如何表示?

\textbf{答:}
TODO

\subsection{IP地址特点}
\textbf{问:}
IP地址的主要特点是什么?

\textbf{答:}
TODO

\subsection{IP地址 vs. MAC地址}
\textbf{问:}
试说明IP地址与MAC地址的区别。为什么要使用这两种不同的地址?

\textbf{答:}
TODO

\subsection{IP数据报首部检验-1}
\textbf{问:}
IP数据报中的首部检验和并不检验数据报中的数据。这样做的最大好处是什么?
坏处是什么?

\textbf{答:}
TODO

\subsection{IP数据报首部检验-2}
\textbf{问:}
当某个路由器发现一个IP 数据报的首部检验和有差错时,为什么采取丢弃的办
法而不是要求源站重传此数据报?计算首部检验和为什么不采用CRC 检验码?

\textbf{答:}
TODO

\subsection{最大传输单元}
\textbf{问:}
什么是最大传送单元MTU? 它和IP数据报首部中的哪个字段有关系?

\textbf{答:}
TODO

\end{document}